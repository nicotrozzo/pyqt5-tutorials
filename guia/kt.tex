\section{Herramienta kt}

\subsection{¿Qu\'e es?}
Supongamos que queremos seguir la estructura de un proyecto de PyQt5 sugerida por esta gu\'ia, \nameref{estructura_programas}, porque resulta pr\'actico
para el manejo de muchos archivos y cuando nos acostumbramos es agradable visualmente. Luego, cada vez que agregamos un Widget, al haberlo creado en QtDesigner debemos
guardarlos en la carpeta designer, y adem\'as compilarlo junto con sus recursos. Y finalmente guardar los resultados de la compilaci\'on en sus respectivas carpetas.
Tambi\'en tenemos que repetir el proceso cada vez que actualizamos o cambiamos algo, esto puede ser un poco tedioso.

Para resolver estos inconvenientes, sub\'i un script para automatizar estos procesos, de forma muy sencilla.

\subsection{Instalaci\'on}
El proceso de instalaci\'on consiste en clonar el repositorio, o simplemente en alg\'un lado de nuestro sistema guardar el script "kt.py", luego
agregarlo a las variables de entorno del sistema operativo para usarlo desde cualquier lado.

\begin{itemize}
    \item 1. Clonamos el repositorio de GitHub.
    \item 2. Agregamos a las variables de entorno la direcci\'on a la carpeta /tools dentro del repositorio. En \nameref{error_de_consola}, se explica como agregar esa variable, pero con Python.
    \item 3. Listo!
\end{itemize}